\chapter{Background and Related Work}
\label{ch:background}

\section{Software Engineering Education}

\subsection{Game-related approaches in SE Education}


\begin{table}[htb]
\caption{Knowledge Areas from \cite{Acm:2015}}
\label{table:knowledgeareas}
\centering
\scriptsize
\begin{tabularx}{\textwidth}{>{\hsize=.3\hsize}X>{\hsize=.7\hsize}X>{\hsize=2\hsize}X}
\hline
\textbf{Acronym} & \textbf{Name} & \multicolumn{1}{c}{\textbf{Description}}  \\
\hline
MAA              & Software Modeling and Analysis          & Modeling and analysis can be considered core concepts in any engineering discipline because they are essential to documenting and evaluating design decisions and alternatives.                                                           \\
REQ              & Requirements Analysis and Specification & The construction of requirements includes elicitation and analysis of stakeholders’ needs and the creation of an appropriate description of desired system behavior and qualities, along with relevant constraints and assumptions.       \\
DES              & Software Design                         & Software design is concerned with issues, techniques, strategies, representations, and patterns used to determine how to implement a component or a system.                                                                               \\
VAV              & Software Verification and Validation    & Software verification and validation uses a variety of techniques to ensure that a software component or system satisfies its requirements and meets stakeholder expectations.                                                            \\
PRO              & Software Process                        & Software process is concerned with providing appropriate and effective structures for the software engineering practices used to develop and maintain software components and systems at the individual, team, and organizational levels. \\
QUA              & Software Quality                        & Software quality is a crosscutting concern, identified as a separate entity to recognize its importance and provide a context for achieving and ensuring quality in all aspects of software engineering practice and process.             \\
PRF              & Professional Practice                   & Professional practice is concerned with the knowledge, skills, and attitudes that software engineers must possess to practice software engineering professionally, responsibly, and ethically.                                            \\
\hline
\end{tabularx}
\end{table}


\subsection{Practical approaches for SE Education}

\section{Project-based Learning - PBL}

One strategy that has been largely used to overcome these challenges is the introduction of software projects in software engineering education (eg.: capstone and project-based courses) (delgado17, marques17). In this context, project-based learning is one of the main successful student-centered pedagogies broadly used in computing science, information systems and engineering courses (marques17, macias12, Jazayeri15, Delgado17, shuto16, warin16, yamada14).

Project-based learning (PBL) is a comprehensive approach to classroom teaching and learning where students are engaged in the investigation of realistic problems, and they learn by working on an open-ended project, discovering problems and finding solutions as they go along. (Blumenfeld91, Jazayeri15). Bender (Bender12) describes PBL as an instructional model based on having students confront real-world issues and problems that they find meaningful, determine how to address them, and then act in a collaborative fashion to create problem solutions. In PBL, the instructor has a less central role, acting as a guide, and students take more responsibility for their own learning, which results in higher student involvement (Martin14, Jazayeri15).

Many software engineering courses use projects as assignments to give students a chance to experience practical problems. However, Jazayeri et al (Jazayeri15) states that this is not enough to implement a PjBL, as these projects are not typically open-ended, and students are expected to follow a very specific path to the solution. Therefore, students are not in charge of learning in these scenarios. In PjBL, projects drive the learning process, and learners should face meaningful problems, where they can investigate, apply and reflect on knowledge and skills useful to solve it. The educator moves from the role of knowledge provider, to the role of facilitator, providing meaningful feedback on students’ actions, helping students to reflect, and providing sufficient guidance for students to achieve learning outcomes.

Gary (Gary15) suggests that in PjBL, “real-worldedness” is achieved by leading students to identify key decision-making factors and draw upon prior experiences. Therefore, while traditional homework and other learning approaches have their significance, PjBL provides more durable benefits, regarding engagement, integration of methods and techniques learned in different courses, and the development of teamwork (Gary15, Warin16). Additionally, sustained interaction over time ensures that students continuously apply and evaluate these experiences (Jazayeri15).

However, there are some challenges related to the use of PBL in SE education. The following subsections address these challenges and issues.

\subsection{PjBL as an Educational Method}

Regarding the adoption of PjBL approaches in SE education, Martin et al (martin14) point issues faced by instructors and by students. Regarding the instructor role, the authors suggests that (martin14): “teachers find difficulties in designing PBL activities that fulfill the main characteristics of this methodology” [PL33], and “Other difficulties are related to the new teachers’ role, where they have to change from a knowledge transmitter to a guide or facilitator, losing control on the student work” [PL35]. Regarding the learner role, the authors points that “In some academic contexts, students are not used to dealing with this kind of problems, therefore they feel lost and end up rejecting the methodology” [PL34].

Warin et al (warin16) mention the lack of comprehensive methodologic frameworks to support educators in setting up courses using this learning method [PL17].  The authors also compare the situation of PjBL and PBL (Problem Based Learning), claiming that while they did not find articles in the literature proposing complete PjBL methods, there are well-established generic methods for PBL that are broadly applicable. Warin et al (warin16) and (macias12) also point the need of tools to support this learning method [P12].

Finally, Yamada et al (yamada14) discuss the issues related to the assessment of the educational effectiveness of this learning method [PL28]. Yamada et al (yamada14) break down this issue as four problems: Obscurity of educational effectiveness, quantitative measurement of the education process, difficulty in quantifying personal characteristics, and difficulty in determining the learning process.

\subsection{Setup of PjBL courses}

The most recurring issue found was that preparing and running PjBL course is time, resource, and effort consuming, both for educators and learners [PL04] (harms16, Hanakawa15, Nguyen13, marques18, Rupakheti17, Daun16, gary15, Mäkiö17). From the instructors perspective, PjBL requires considerable effort to supervise, guide and mentor students over a significant period of time (gary15, Daun16). There is also investment in setting up proper projects for students to tackle (Rupakheti17). For students, PjBL ensures sustained, long-term participation, what is contrary to bursty nature of students (gary15). However, they may not be able to devote the necessary time and effort because they may have other subjects to study at the university (hanakawa15).

Other problems that are directly associated to PL04 are: PjBL requires some level of preparations [PL22], as instructors are required to produce lecture material, academic examples, or project milestones (Daun16); instructors may need to specify a process that allows students to reach their learning outcomes [PL14]; some projects require setting up adequate development environments [PL02]  and physical space [PL24]; finally, some learning outcomes may require proper selection of state of art tools that student should use during the execution of their projects [PL16]. 
All those issues impact on scalability [PL03]. Scaling PBL is difficult both in terms of student head count and integration across the degree program (gary15). For instance, Harms et al (Harms16) claims that there is a limit to the number of student-led projects that an instructor can manage, and in their experience, using student-led projects with a class of more than 30 students was hard to manage.

\subsection{Selection of meaningful projects}

In PjBL, projects play a central role in the learning process. Therefore, the selection of adequate projects for students’ immersion is a crucial step in the setup of PjBL courses. The selection of projects that provide good balance in size, complexity and type of stakeholders [PL05] is a challenge that educators are exposed to. Harms et al (harms16) states that “projects need to not be too shallow and yet not be too idealistic either”.

As a consequence, there is a discussion on the use real versus realistic projects, i.e. the use of real projects from the community or industry, where real customers act as stakeholders, versus the use of projects that simulate real problems, where instructors play the role of customers or problem providers. In the former, educators must face the problems of the difficulty in establishing partnerships with the industry [PL20], as the immediate rewards and time availability for industry practitioners to collaborate in instruction are quite limited, there might be restrictions related to intellectual property, and there is no guarantee for successful outcomes for companies (daun16). Additionally, the participation of external person as real customers may be problematic [PL30], as their availability and expectations are not controlled.

In the case of educators acting as stakeholders [PL21], Daun et al (daun16) warns:
“(…) this requires a high sensitivity and  experience of the instructors regarding their ability to foresee student challenges, maintain the acted role throughout the semester, and carefully guide the knowledge discovery process in such a way that the students achieve teaching goals and perform satisfactorily. Moreover, this also requires a large amount of in-depth industry knowledge, which unless the instructor has an industry background, may not be attainable. Furthermore, the instructor will continuously have to switch between both rolls (i.e. stakeholder and teacher). Hence, there is a threat that students will take statements given by the stakeholder falsely as instruction or solution given by the teacher.” (daun16).

Nguyen et al (Nguyen13) described a problem regarding the use of real problems, where students who worked on small components of very large-scale projects in some IT companies. The authors claim these students did not have a big picture of what they were working on, and thus did not realize much benefit of their IT skills and knowledge. Therefore, the choice of projects may impact in the coverage of expected learning outcomes [PL10]. However, we believe that this problem is not specific to the context of real projects, as we discuss it further in section IV-D.

Finally, it is important that the projects are chosen considering aspects such as the relevancy of the application domain [PL15]. The choice of application domain must balance student excitement and relevancy for industry, however, this is a hard choice, as relevancy in this context may be volatile trends (delgado17). 

It is important to consider that PjBL deals with ill-structured problems as central activities [PL21], which Martin et al (martin14) consider is one of the cornerstones of PBL and one of the main sources of difficulties in using PBL. In ill-structured problems, one or more of the problem elements are unknown or not known with any degree of confidence, goals are vague or unclear, there are multiple solutions and solution paths (or even no consensual solution), they present uncertainly about which concepts, rules and principles are necessary, learners are required to express personal opinion, beliefs or judgments. This may lead back to problem PL34.

\subsection{Tracking students progress and learning outcomes}

Considering the nature of PjBL activities, there is the issue of requiring the right amount of guidance towards learning outcomes [PL25]. Considering the open-ended nature of the activities, no two projects or project teams are the same, so each student has different experience (gary15). However, the learning outcomes remains the same for everyone, and instructors have to keep close attention on keeping students on track of the learning goals. 

Considering problems PL15 and PL25, instructors are required to spend considerable effort in tracking students progress and providing meaningful feedback for students to ensure learning [PL01]. Tracking the progress of students projects is not easy, as instructors are required to scrutinize the report by students or participate in projects in order to understand progress of the projects, what forces educators to bear the burden (Fukuyasu13). Instructors also have to be aware of the risk of students unethically using source code of others and claiming its their own work [PL06]. 

Consequently, the definition of a strategy to evaluate students is also difficult [PL18], as it may require considering qualitative and quantitative data on their performance. If not clearly defined, students may become confused and concerned regarding their assessment [PL23]. As students direct their activities based on the given assessment criteria [11], the assessment design plays a key role in what students will focus on [Fagerholm13]. 

Educators must consider that the projects should be aligned with expected learning outcomes (as discussed on in PL10 in previous section). In technical domains, such as mathematics or computer science, it may be difficult to instruct a variety of largely independent but closely related concepts and topics (daun16). Therefore, a challenge arises when dealing with instructing instruct complex theoretical relationships in a sound fashion [PL19]. In software engineering, for instance, software process is a transversal knowledge, and each of the phases of software development life cycle, is closely related to the other, however each one has a multitude of learning topics. It is difficult to impart the importance of each topic and the impact they have on each other.

Additionally, students may become absorbed by a single facet of the project, such as programming [PL08]. Hanakawa et al (hanakawa15) illustrate this matter in their first attempts to introduce contest-based learning, where the students became absorbed in programming rather than design and analysis activities. Similarly, Winterfeldt14 describe their experiences using a project-based approach of teaching application design. However, they recognized as a general challenge that students focus on details of an implementation (technology and coding) rather than on the application structure (Architecture and design).

Finally, when students have to interact with new technologies, they may get overwhelmed when searching for relevant instructional materials [PL13], 


\subsection{Teamwork and different types of learning}

Although PjBL does not necessarily involves team activities, it has been largely used in software engineering education to support practicing soft skills such as teamwork, communication, leadership, and collaboration. Specific topics related to software process or software project management also benefit from projects executed in teams. Therefore, some authors suggest that instructors should plan carefully about team composition [PL26], as teams composed by members with the right complementary skills may improve the learning experience, while others compositions may compromise it (sunaga16, sunaga17, yamada14).

Besides that, students working in teams may be affected by their lack of teamwork experience [PL11]. Chen14 state that “if not equipped with the necessary teamwork skills, these students are like blind explorers trying to find the proper direction in which to take their project”. The authors support that the lack of team experience impedes learning and makes it difficult to obtain quality results. 
Kizaki14 mentions other problems related to teamwork: shortage of communication between members [PL29] and difference of a member's technical capabilities [PL31]. The later, may also impact in the problem of a heterogeneous effort distribution among team members [PL32], what may lead to knowledge not being equally distributed [PL09]. As a consequence, it becomes even harder to individually assess students progression toward learning outcomes, due “to the ingenuity some students show in hiding behind others’ work” (gary15).

Finally, students are impacted differently by the teaching approach, as they may have different learning styles (zhi16). Therefore, a challenge using PjBL is to address different learning styles of students [PL07], which may require mixing different learning environments, resources, modes and/or contents.



\section{Gamification}

\section{Related Work}

\section{Final Remarks}