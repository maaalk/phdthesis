\chapter{Background and Related Work}
\label{ch:background}

\section{Software Engineering Education}

\subsection{Game-related approaches in SE Education}


\begin{table}[htb]
\caption{Knowledge Areas from \cite{Acm:2015}}
\label{table:knowledgeareas}
\centering
\scriptsize
\begin{tabularx}{\textwidth}{>{\hsize=.3\hsize}X>{\hsize=.7\hsize}X>{\hsize=2\hsize}X}
\hline
\textbf{Acronym} & \textbf{Name} & \multicolumn{1}{c}{\textbf{Description}}  \\
\hline
MAA              & Software Modeling and Analysis          & Modeling and analysis can be considered core concepts in any engineering discipline because they are essential to documenting and evaluating design decisions and alternatives.                                                           \\
REQ              & Requirements Analysis and Specification & The construction of requirements includes elicitation and analysis of stakeholders’ needs and the creation of an appropriate description of desired system behavior and qualities, along with relevant constraints and assumptions.       \\
DES              & Software Design                         & Software design is concerned with issues, techniques, strategies, representations, and patterns used to determine how to implement a component or a system.                                                                               \\
VAV              & Software Verification and Validation    & Software verification and validation uses a variety of techniques to ensure that a software component or system satisfies its requirements and meets stakeholder expectations.                                                            \\
PRO              & Software Process                        & Software process is concerned with providing appropriate and effective structures for the software engineering practices used to develop and maintain software components and systems at the individual, team, and organizational levels. \\
QUA              & Software Quality                        & Software quality is a crosscutting concern, identified as a separate entity to recognize its importance and provide a context for achieving and ensuring quality in all aspects of software engineering practice and process.             \\
PRF              & Professional Practice                   & Professional practice is concerned with the knowledge, skills, and attitudes that software engineers must possess to practice software engineering professionally, responsibly, and ethically.                                            \\
\hline
\end{tabularx}
\end{table}


\subsection{Practical approaches for SE Education}

\section{Project-based Learning - PBL}

Project-based learning (PBL) is a comprehensive approach to classroom teaching and learning where students are engaged in the investigation of realistic problems, and they learn by working on an open-ended project, discovering problems and finding solutions as they go along. (Blumenfeld91, Jazayeri15). Bender (Bender12) describes PBL as an instructional model based on having students confront real-world issues and problems that they find meaningful, determine how to address them, and then act in a collaborative fashion to create problem solutions. In PBL, the instructor has a less central role, acting as a guide, and students take more responsibility for their own learning, which results in higher student involvement (Martin14, Jazayeri15).


\section{Gamification}

\section{Related Work}

\section{Final Remarks}