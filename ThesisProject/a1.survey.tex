\chapter{Games and Gamification in Software Engineering Education: A Survey with Educators}
\label{ch:survey}

This chapter presents the results of a survey study with Software Engineering (SE) educators, in order to to understand the level of adoption of Game-based Learning and Gamification to support SE education. To achieve this goal, we analyze the results of a questionnaire sent to 285 SE professors from 100 renowned superior education institutions in our country. As a result, 88 professors participated in the study. The results of this study have been submitted to Frontiers in Education, and is currently under review process.

This chapter is organized as follows. Section \ref{sec:studySettings} describes the study design, research questions and methods. Section \ref{sec:goals} presents the results of the study. Section V discusses the main findings and implications of the study. Section VI discusses the possible bias that threatens the validity of this study. Section VII presents the related work. Finally, Section VII presents our conclusions and future works.

\section{Study Settings}
\label{sec:studysettings}
This section explains how we planned and executed this study.

\subsection{Study Goal and Research Questions}
\label{sec:goals}

The goal of this study is to investigate the adoption of Serious Games and Gamification in the context of SE Education from the perspectives of professors of renowned higher education institutions in Brazil. To achieve this goal, we formulated three Research Questions (RQ) presented below:

\begin{itemize}
\item \textbf{RQ1.} Do SE professors use Game-based Learning to support SE Education? 

\item \textbf{RQ2.}  Do SE professors use Gamification to support SE Education?

\item \textbf{RQ3.}  What are the reasons for not adopting Serious Games and Gamification in SE Education?
\end{itemize}

Therefore, for RQ1 and RQ2 our interest is (i) to collect information about the use of games and gamification in classrooms by Software Engineering professors and (ii) to understand the relation of ACM/IEEE knowledge areas and the game-related methods. For RQ3 our goal is to understand what reasons avert educators from using these approaches.

\subsection{Study Design and Research Methods}
\label{sec:studydesign}

To answer the research questions, we performed an Opinion Survey study. According to Easterbrook et al. [10], Survey studies are used to identify characteristics of a wide population and are usually associated to the application of questionnaires. Surveys are meant to collect data to describe, compare or explain knowledge, attitudes and behaviors [11].
Therefore, the following steps were defined for the execution of this study: (i) Definition of study goals and research questions (Section \ref{sec:goals}); (ii) Creation of the questionnaire (Section \ref{sec:questionnaire}); (iii) Execution of a pilot study to validate the questionnaire (Section III-C); (iv) Invitation of the target population to participate in the study (Section III-D); (v) Data normalization, analysis and report of results (Section IV).

\subsection{Questionnaire}
\label{sec:questionnaire}

To answer the research questions, we created an electronic survey using the Google Forms tool. The survey had five sections: (1) general information about the participants, (2) SE education, (3) the utilization of game-based approach in SE education, (4) the utilization of Gamification in SE education and (5) the reasons of professors for not using the games and Gamification in SE education. We used multiple-choice and open questions in the questionnaire.

We performed a pilot study with 10 SE professors. From the feedbacks of the pilot study, only minor corrections were made. Tables I, II, III, IV and V present the questions in the final version of the questionnaire. Table I presents the questions about the participants background.

TABLE I. BACKGROUND QUESTIONS.


TABLE II. QUESTIONS ABOUT THE PARTICIPANT EXPERIENCE IN SE EDUCATION


Table II presents the questions about the Software Engineering teaching. In Q4, by defining the options “Software Engineering 1” (SE 1) and “Software Engineering 2” (SE 2) we attempted to condensate the many variable names of national Software Engineering disciplines and separate them in the basic and advanced options. The pilot study and the final results show that these options were understood by the participants. In Q7, the areas shown for the participants are an adaptation of the areas defined by ACM and IEEE curriculum guidelines [1].

Table III presents the questions related to the utilization of game-based approach in Software Engineering teaching. The questions Q9, Q10, Q11 and Q12 were only reached by the participants that answered “Yes, and I have already used in my classrooms” in Q8.

TABLE III. QUESTIONS ABOUT THE USE OF GAMES IN SE EDUCATION.


Table IV presents the questions related to the utilization of Gamification approach in Software Engineering teaching. The questions Q14, Q15 and Q16 were only reached by the participants that answered “Yes, and I have already used in my classrooms” in Q13.
Finally, Table V presents questions on the reasons of professors for not using the games and Gamification in SE education.


TABLE IV. QUESTIONS ABOUT THE USE OF GAMIFICATION IN SE EDUCATION

TABLE V. QUESTIONS ABOUT THE REASONS FOR NOT USING THE APPROACHES AND THE PERSPECTIVE OF FUTURE USE.

\subsection{Population and Sampling}
\label{sec:population}

The target population of this study are SE professors from higher education institutions in our country. To select an appropriate sample of this population, we used a ranking of the top one hundred well-established universities and educational institutions of different regions of our country, which is updated yearly by a relevant magazine company. We searched the websites of each institution for the professors associated to departments or undergraduate programs related to Software Engineering, Computer Science and alike. Finally, we searched the Lattes Platform [12], which is the official curriculum platform for higher education professors and researchers in our country, and then we filtered the professionals relevant to our sampling. This step was very important to guarantee that all invitation e-mails were sent to active Software Engineering professor.

For each contact gathered, we sent a personalized invitation e-mail to participate in our study. The idea of the personalized email was for the recipient not confusing the invitation with spam and marketing e-mails, and for the participant to notice he was objectively picked because of his relevance in our sample. This particular strategy worked very well: for the 285 invitations sent, we had 88 answers in our survey (30.9\%). This participation rate exceeded our expectations, considering that the target population was very specific. Therefore, we believe our sample is significant for our target population.

\section{Results}

From 285 invitations sent, we collected 88 responses. All of the 88 responses were validated and entered in the final results analysis. The sample was composed of professors from 45 different institutions. There are 60 male participants (68.2\%) and 28 female participants (31.8\%). Regarding the age of the participants, 7 (8\%) are aged up to 30 years old, 64 (72.7\%) between 31 and 50, and 17 (19.3\%) are more than 50 years old. Finally, regarding their experience in teaching SE, 37 participants (42\%) teach SE from 5 to 15 years, 28 (31.8\%) for more than 15 years, and 23 (26.1\%) for less than five years.

Figure 1 shows the distribution of the answers regarding the disciplines taught by the participants (Q4). Considering that Software Engineering is a very diverse and dense area in Computer Science, we tried to resume Software Engineering in two basic categories, one for basic and other for advanced subtopics: SE 1 (introduction) and SE 2 (advanced topics). In addition, we included the option “Specific disciplines”, that aimed subtopics of Software Engineering explored to a level where a full discipline was created and used to teach this content. Seventy-one participants (80\%) declared that they teach “Specific disciplines”, 55 participants (62.5\%) declared that teach SE 2, and 34 participants (38.6\%) declared that teach SE1. 
 
Fig. 1. Disciplines taught by the participants

Table VI shows the most recurrent specific disciplines cited in the answers of Q5. It was an open field, so, in many cases, multiple disciplines were listed. Other 35 unique specific disciplines were cited, what shows the diversity and density of Software Engineering.

TABLE VI. SPECIFIC DISCIPLINE MENTIONED IN Q5.

Figure 2 shows the coverage of the SE knowledge areas (Q7), where the values 1, 2 and 3, respectively represent “I don't cover this area”, “I cover superficially this area” and “I extensively cover this area”.  By far, “Professional practices” is the least covered area. “Software process” and “Software modeling” and “Software requirements” are the top 3 covered areas.
 
Fig. 2. SE knowledge areas covered by the participants.

\subsection{The use of Game-based Learning in SE education (RQ1)}
	
Regarding the participants knowledge and use of Game-based learning (Q8), 54 participants (61.4\%) know the approach but never used it, 21 participants (23.9\%) have already used this approach, and only 13 participants (14.8\%) have no knowledge on the subject.

The participants that claimed to have used Game-based Learning mentioned 24 games as responses for Q9. Only SimSE [7] and UbiRE [13] had more than one occurrence. Other examples of games include SimSE SPIAL, Code Defenders, Dojo, Sesam, u-Test, JoVeTest, InspSoft, iTestLearning, ScrumGame, XPGame, Requirement Island, Airplane Factory, and other games with generic names such as “Elicitation Theater” and “Puzzle”. Figure 3 shows the SE knowledge areas the participants used games to teach (Q10). Project Management, Software Process and Software Requirements are the top 3 knowledge areas covered with games, while Software Architecture, Software Maintenance and Software Modelling are the least covered topics.
 
Fig. 3. SE knowledge areas covered by the games

The participants were asked to rate the acceptance of the approach by students and the overall success of its use in their classroom (Q11). This impression was asked in a grade form, where 0 meant “Really bad acceptance, many problems” and 5 meant “Great acceptance, no problems”. Figure 4 shows there was an overall positive response for this question, with only grades higher or equal to 3.
 
Fig. 4. Reception of Game-based Learning by the students in the perspective of the participants

Finally, the main benefits and difficulties observed in the use of Game-based learning (Q12) are listed in Tables VII and VIII respectively. The benefits were related to keeping students interested, engaged and motivated, improvement in students understanding and assimilation of contents, increased student participation, increased grades and the novelty of the approach. There was no consensus on the difficulties of using Game-based Learning, and 12 different issues were pointed by the participants.

TABLE VII. BENEFITS OBSERVED FROM THE USE OF GAME-BASED LEARNING IN SE EDUCATION.

\subsection{The use of Gamification in SE education (RQ2)}

Regarding the participants knowledge and use of Gamification in SE education (Q13), similar to the responses of Q8 (Section IV-A), the majority of the participants (56 participants – 63.6\%) know the approach but never used it, 19 participants (21.6\%) have already used this approach, and only 13 participants (14.8\%) have no knowledge on the subject.

Table IX presents the elements of games used by professors that claimed to have used game elements or Gamification (Q14). Points, Levels and Quizzes were the most mentioned game elements. However, Contests, Badges, Levels and Leaderboards are also mentioned significantly. The literature on Gamification in SE suggest that Points, Levels and Leaderboards are the most common elements used [2].

TABLE IX. GAME ELEMENTS MENTIONED IN THE ANSWERS OF Q14.

Figure 5 shows the distribution of responses for Q15. The impressions of the 19 participants who have used this approach regarding the students’ acceptance of this approach was positive. However, there were two participants who answered the values “0” and “1”. Therefore, the success of this approach was not a consensus.
 
Fig. 5. Reception of Gamification by the students in the perspective of the participants

Finally, Tables X and XI presents the benefits and difficulties of using Gamification in SE education, respectively (Q16). Improving students’ interest in learning is the dominant benefit described by the participants. For the difficulties, the main issue is objectively measuring any improvement in performance from the students. Four issues are related to the difficulties in executing the approach (higher effort for the professor, lack of tool support, difficulty to adapt the approach to the learning process, and the difficulty in providing instant feedback). The other two problems were related to the students’ reception.


TABLE X. BENEFITS OBSERVED FROM THE USE OF GAMIFICATION IN SE EDUCATION

\subsection{Reasons for not using Game-based Learning and Gamification in SE education}

The participants who responded that had never used the approaches (Q8 and Q13) were asked the reasons for not using these approaches (Q17). Table XII list the results for this question. The lack of knowledge about the mentioned approaches is the most recurring reason pointed. The answers related to “Lack of knowledge of appropriate games”, “Lack of materials” and “Lack of resources” correspond to 20 out of 63 (31.7\%) of the reasons, and these answers may be related to the difficulty in finding relevant guidelines or centralized resources to support the use of these approaches.

TABLE XII. REASONS FOR NOT USING GAME-BASED LEARNING AND GAMIFICATION IN SE EDUCATION.

Finally, the participants were asked if they would consider using these approaches in the future (Q18). In our sample, only 12 participants stated that had already used both approaches. Removing these participants, we have a total of 76 responses. In this group, 35 participants (46.1\%) were positive about using these approaches in the future, 30 participants (39.5\%) were in doubt, and 7 participants (9.2\%) stated that would not use. Four participants (5.3\%) did not answer the question. The 7 participants that stated that do not consider using these approaches in the future have never used neither games nor game elements. Therefore, there is a positive tendency towards the adoption of games and game elements in SE education.

\section{Discussion}
\label{sec:discussion}

In this section we discuss some relevant findings and issues observed in the results of Section IV, and we revisit the Research Questions defined in Section \ref{sec:goals}.

\subsection{Do SE professors use Game-based Learning to support SE Education? (RQ1)}

Our study shows that a significant amount of SE professors uses games to support SE Education. Seventy-five participants (85.2\%) are aware of this method, and 21 (28\%) have already used it. However, considering that games for SE education is not novelty [2], it is surprising to observe that 13 participants (14.2\%) had no knowledge about this educational approach.

Regarding the purpose of the games for the participants who have used them, our results show that the knowledge areas of “Project Management”, “Software Process”, and “Software Requirements” were the most mentioned as topics covered by the use of games.  These topics have in common the difficulty in providing meaningful examples relying only in theory or are hard to simulate in educational projects. Oliveira et al. [14] suggests that “traditional approaches usually adapt and simplify problems, thus, reducing their relevance” and that “Problems are also usually linked to prefabricated solutions that do not help them to develop their own ideas to tackle problems”.

In general, the participants believed the use of games in SE education has been well received by their students. The most recurrent benefits noted by the professors were: higher students interest (7), higher engagement (4), higher motivation (2) and higher content assimilation (2). This result indicates that the students liked the new approach when used and it was clearly noted by their professors. It seems that the common explanatory teaching methodology tends to generate monotony and disinterest over time, and new and different approaches, when used right and in the appropriate moment, renews the students’ attention and interaction with the subject, leading to a better content assimilation.

There was no consensus regarding the difficulties in applying the methodology. We identified that some of these issues could be overcome with more experience by the professors, like “Extra workload for professor”, “Games need a specific learn rhythm” and “Implementation in classroom”. Other issues seems to be related to the game chosen: “Language”, “Game comprehension” and “Games are old and outdated”. Again, statements that leads to the problem of choosing a game to apply the methodology.


\subsection{Do SE professors use Gamification in SE Education?}

Our study shows that a substantial number of participants have already used game elements to support SE Education. Out of the 75 professors that did knew this methodology, 19 (25.3\%) already used in classes.  Additionally, considering that only 12 participants have used both approaches, this number is expressive, since Gamification is much more recent than the use of Serious Games in SE education [2].

The most popular game elements among the participants were: Points (15), Challenges (13), Quizzes (11), Badges (7), Milestones (7), Levels (7) and Contests (7) were expected due to Anonymous (2018) study. Some participants even explained how those elements were used: 

“In my disciplines, for example, i have being giving the opportunity to the students to exchange quizzes points (fast exercises lists) for deadline extension days (in a limited way, of course)”

The participants also perceived a positive reception of the usage of this methodology. Almost 80\% of the ratings were 4 or 5, where 5 was the maximum representing “Great acceptance, no problems” by the students. Rate 4 represented 57.9\% of the responses. In this context we had only one occurrence of rate 0 and one of rate 1. One of these bad ratings were related to bad application of the methodology. For instance, the participant who rated 1 stated:

“I had difficulties from not having an integrated environment with Moodle (students and professors platform) (...). We ended having the elements described in classrooms manually, what wasn’t attractive”

The most recurrent benefits noticed from the use of game elements was “higher student interest” (8). Regarding the difficulties, the most pointed out was the “Difficulty in measuring performance improvements” (3 occurrences). 
C.	What are the reasons for not adopting Serious Games and Gamification in SE Education? (RQ3)
   There is some kind of consensus in the responses for not adopting serious games and gamification in SE Education. The occurrences were: lack of knowledge about the approaches (19), lack of knowledge about fitting games (14), lack of time (12), don't believe in the approaches (6), lack of interest (5), lack of material (4), lack of resources (2) and difficulty in correct use by students (1). However, further investigation is required to understand specific motivations for different professors and understand possible correlations.
One particular case that came to our attention, for instance, was that all participants under 30 years old are aware of the use of games for teaching SE. However, when we consider the years of experience in teaching SE, only 13% of the participants with least experience (0 to 5 years) have used games in classroom, while this percentage is 29% and 25% for the groups with 5 to 15, and more than fifteen years of experience, respectively. 
One possible cause is that less experienced professors may not have time, space or confidence to use these approaches. In their responses, the reasons they alleged for not using games in classroom were most related to the lack of better understanding or the approach, not finding the adequate games that fit in their classrooms, and the investment (time, resources and effort) required to adapt their classroom to introduce games. Other possible cause is that it is time and effort demanding to find, understand and adapt educational SE games for the use in classroom. 
However, the responses for question Q18 shows that there is a positive tendency toward using these game-related approaches in the future. Even though there are many scientific papers proposing and evaluating game related approaches for SE education, it seems that there is still a problem in disseminating these approaches among professors and providing instructors with relevant resources to facilitate the adoption of these tools. Therefore, it would be relevant for the community to strive for centralized repositories of information about game-related methods for SE education.
VI.	THREATS TO VALIDITY
In this section, we document potential threats to the study validity and discuss some bias that may have affected the study results. We also explain our actions to mitigate them.
Results: The results presented in the study reflect our interpretation of the data collected from the answers of the questionnaire. However, the questions were objective enough to enable readers to derive their own interpretations. In addition, there may be several other important issues in the data collected, not yet discovered or reported by us. 
Population Sample: Considering the specific population of SE professors in our country, we believe that our procedure for identifying and inviting relevant participants retrieved a considerable number of candidates for the study (295). However, even considering the participation rate achieved in this study (30.9%), our results may not reflect the opinion of the general community of SE educators. Another limitation is that our sample was limited to the SE professionals of our country, therefore, it is not possible to generalize the results to the global community of SE professors.
Questionnaire: The validity of the questionnaire may be threatened by ambiguous questions that may compromise the answers of the participants. However, we executed a pilot study with 10 SE educators to identify possible problems related to ambiguity, missing response options, and lack of clarity in our questions. Additionally, the questionnaire was created by three researchers, two having experience in SE education and research. 
VII.	RELATED WORK
This section discusses other studies that are somehow related to the present research described in this paper. We consider related works, studies that investigate the state-of-the-practice in the adoption of Game-based Learning and Gamification in SE education. We found one study specifically surveying SE instructors regarding the use of games in SE education [15]. Most of the related work we found are focused in surveying the literature in order to map the existent game-related approaches in SE education [1, 16, 17].
Albayrak [15] investigates the factors that influence the instructor’s acceptance of the utilization of games (whether serious or not) in undergraduate software engineering education. In this study, a survey was performed with 30 faculty members teaching SE in turkey. Their results show that “the number of hours per week the instructor plays game”, “instructor’s experience in using games for educational purposes in general”, and “instructor’s experience in designing games” have significant impact on the instructor’s decision to use games in software engineering education. Our results provide a different perspective, aiming on why instructors do not use games in SE education. However, the results are convergent in relation to the fact that professors experience and knowledge on the use of games for teaching SE is a key factor for using them or not.
In relation to the purpose of using game-related approaches in SE education, our results are in accordance to the results of the secondary studies of Anonymous [2] and Cautifield et al. [16]. Anonymous [2]} mapped game-related approaches to the knowledge areas of the curricular guidelines of ACM/IEEE [1] and found that the knowledge areas “Software Process” (which also include topics related to project management) and “Software Requirements” were the most covered by those approaches. Cautifield et al. [16] mapped 36 studies on the use of games and simulations for SE education to SWEBOK areas. Their results show that these games and simulations have been mostly used to cover software engineering management and development processes areas.
Regarding Gamification, Alhammad and Moreno [17] identified that the most positively affected aspect was student engagement. In our results, the students’ interest was the most positive benefit observed by our participants who used Gamification in their classrooms. The authors also observed that points and leaderboards were the most frequently adopted gaming elements individually and together and same applies to the challenges and feedback mechanics. Similarly, in our results we observed points and challenges as the most recurrent game elements.
VIII.	CONCLUSION
This paper presented the results of a survey with SE professors to investigate the adoption of Game-based Learning and Gamification in SE education. We invited 285 professors from 100 renowned higher education institutions in our country to participate in the study. A total of 88 participants answered a questionnaire with 18 questions. The questions were related to three research questions: (RQ1) Do professors use Game-based Learning to support SE Education? (RQ2) Do professors use Gamification to support SE Education? (RQ3) What are the reasons for not adopting Serious Games and Gamification in SE Education?
The results show that both approaches are known by most of the participants, however only a fraction of them have already applied these approaches. Serious games have been most used to cover “Software Process”, “Project Management” and “Software Requirements” knowledge area, while “Software Architecture”, “Software Maintenance” and “Software Modelling” are the least covered topics. In the case of Gamification, the most used game elements are Points, Quizzes, and Challenges. The main reasons for not using these approaches are related to lack of knowledge, lack of information about relevant games for teaching SE, and the lack of time to plan and include these approaches in the classroom. Finally, there is a positive tendency toward the future adoption of these game-related approaches by the participants.
A future work is the creation of a web platform for maintaining and promoting serious games for software engineering, related materials, and resources to support the adoption of game-related approaches in SE education. This is intended to facilitate educators interested in the adoption of such approaches.
REFERENCES
[1]	IEEE & ACM JTFCC, "Software Engineering 2014: Curriculum Guidelines for Undergraduate Degree Programs in Software Engineering," in IEEE & ACM;The Joint Task Force on Computing Curricula, November 23 2015.
[2]	Anonymous, 2018. REFERENCE OMMITED FOR BLIND REVIEW
[3]	S. Deterding and D. Dixon, “Gamification: Using game design elements in non-gaming contexts”. In Extended Abstracts on Human Factors in Computing Systems (CHI). 2011.
[4]	M. R. Marques, A. Quispe and F. S. Ochoa, "A systematic mapping study on practical approaches to teaching software engineering," in Frontiers in Education Conference (FIE), 2014. 
[5]	C. G. V. Wangenheim and F. Shull, "To game or not to game?," IEEE Software, vol. 26, no. 2, pp. 92-94, 2009.
[6]	A. Baker, E. O. Navarro and A. v. d. Hoek, "Problems and Programmers: an educational software engineering card game," in 25th International Conference on Software Engineering (ICSE), pp. 614–619., 2003. 
[7]	E. Navarro and A. Van Der Hoek, "Multi-site evaluation of SimSE," in ACM SIGCSE Bulletin. Vol. 41. No. 1, 2009.
[8]	B. S. Akpolat and W. Slany, “Enhancing software engineering student team engagement in a high-intensity extreme programming course using gamification”, IEEE Conference on Software Engineering Education and Training (CSEE&T), 2014
[9]	V. Uskov and B. Sekar, “Gamification of software engineering curriculum”, Frontiers In Education Conference (FIE), 2014
[10]	S. Easterbrook, J. Singer, M. A. Storey and D. Damian, "Selecting Empirical Methods for Software Engineering Research". Guide to advanced empirical software engineering. Springer-Verlag. pp.285-311, 2008.
[11]	S. Pfleeger and B. Kitchenham, “Principles of survey research: part 1: turning lemons into lemonade”. SIGSOFT Software. Engineering Notes, 26, 6 (November 2001), 16-18. 2001
[12]	REFERENCE OMMITED FOR BLIND REVIEW
[13]	T. Lima, B. Campos, R. Santos and C. Werner, “UbiRE: A game for teaching requirements in the context of ubiquitous systems”. In XXXVIII Conferencia Latinoamericana En Informatica (CLEI). Medellin, Colombia. 2012.
[14]	C. Oliveira, M. Cintra and F. Mendes Neto, "Learning risk management in software projects with a serious game based on intelligent agents and fuzzy systems," in 8th conference of the European Society for Fuzzy Logic and Technology (EUSFLAT), 2013.
[15]	O. Albayrak, “Instructor's Acceptance of Games Utilization in Undergraduate Software Engineering Education: A Pilot Study in Turkey”. IEEE/ACM 4th International Workshop on Games and Software Engineering (GAS), 2015.
[16]	C. Caulfield, J. C. Xia, D. Veal and S. P. Maj, "A systematic survey of games used for software engineering education," Modern Applied Science, 5(6), p. 28, 2011.
[17]	M.l M. Alhammad and A. M. Moreno, “Gamification in software engineering education: A systematic mapping”. In Journal of Systems and Software, Vol. 141, pp. 131–150. 2018. 


 


