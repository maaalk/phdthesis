\chapter{Primary Studies used in the Systematic Mapping}
\label{ch:primarystudies}

The selection process described in Chapter \ref{ch:sms} resulted in 156 primary studies, listed as follows:

[PS001]	B. S. Akpolat and W. Slany, “Enhancing software engineering student team engagement in a high-intensity extreme programming course using gamification”, IEEE Conference on Software Engineering Education and Training, 2014


[PS002]	A. L. D. Buisman and M. C. J. D.Van Eekelen, “Gamification in educational software development”, Computer Science Education Research Conference, 2014


[PS003]	K. Berkling and C. Thomas, “Gamification of a software engineering course and a detailed analysis of the factors that lead to it's failure”, International Conference on Interactive Collaborative Learning, 2013


[PS004]	V. Uskov and B. Sekar, “Gamification of software engineering curriculum”, Frontiers In Education Conference, 2014

[PS005]	M. Laskowski, “Implementing gamification techniques into university study path - A case study”, IEEE Global Engineering Education Conference, 2015


[PS006]	L. Singer and K. Schneider “It was a bit of a race: Gamification of version control”, International Workshop on Games and Software Engineering, 2012


[PS007]	J.N. Long and L. S. Young, “Multiplayer on-line role playing game style grading in a project based software engineering technology capstone sequence”, ASEE Annual Conference and Exposition, 2011


[PS008]	C.Thomas and K. Berkling, “Redesign of a gamified Software Engineering course”, International Conference on Interactive Collaborative Learning, 2013

[PS009]	W. Q. Qu, Y. F. Zhao, M. Wang and B. Liu, “Research on teaching gamification of software engineering”, International Conference on Computer Science Education, 2014

[PS010]	J. Bell, S. Sheth and G. Kaiser, “Secret Ninja Testing with HALO Software Engineering”, International Workshop on Social Software Engineering, 2011

[PS011]	C.-H.a Su and C.-H.b Cheng, “3D game-based learning system for improving learning achievement in software engineering curriculum”, Turkish Online Journal of Educational Technology, 2013

[PS012]	J. Srinivasan and K. Lundqvist, "A constructivist approach to teaching software processes", International Conference on Software Engineering, 2007

[PS013]	E. Knauss, K. Schneider and K. Stapel, "A Game for Taking Requirements Engineering More Seriously", International Workshop on Multimedia and Enjoyable Requirements Engineering, 2008

[PS014]	L. Ohlsson and C. Johansson, "A Practice Driven Approach to Software Engineering Education", IEEE Transactions on Education, 1995

[PS015]	S. Zuppiroli, P. Ciancarini and M. Gabbrielli, "A role-playing game for a software engineering lab: Developing a product line", IEEE Conference on Software Engineering Education and Training, 2012

[PS016]	T. A. B. Galvao, F. M. M. Neto, M. F. Bonates and M. T. Campos "A serious game for supporting training in risk management through project-based learning", Communications in Computer and Information Science, 2012

[PS017]	T. Wang and Q. Zhu, "A software engineering education game in a 3-D online virtual environment", International Workshop on Education Technology and Computer Science, 2009

[PS018]	A. Hoffmann, "A Trainer's Guideline to Teaching Soft Skills Using Improvisation Theater: A Workshop Format Exemplified on a Requirements Engineering Game", European Conference on Pattern Languages of Programs, 2012

[PS019]	T. D. Lynch, M. Herold, J. Bolinger, S. Deshpande, T. Bihari, J. Ramanathan and R. Ramnath, "An agile boot camp: Using a LEGO®-based active game to ground agile development principles", Frontiers In Education Conference, 2011

[PS020]	T. M. Connolly, M. Stansfield and T. Hainey, "An application of games-based learning within software engineering", British Journal of Educational Technology, 2007

[PS021] 	A. Baker, E. O. Navarro and A. van der Hoek, "An experimental card game for teaching software engineering", IEEE Conference on Software Engineering Education and Training, 2003

[PS022] 	A. Baker, E. Oh Navarro and A. Van Der Hoek "An experimental card game for teaching software engineering processes", Journal of Systems and Software, 2005

[PS023]	A. Y. K. Chua and R. S. Balkunje, "An exploratory study of game-based m-learning for software project management", Journal of Universal Computer Science, 2012

[PS024]	R. Atal and A. A. Sureka "A software engineering simulation game for teaching practical decision making in peer code review", CEUR Workshop Proceedings, 2015

[PS025]	G. Rong, He Zhang and D. Shao, "Applying competitive bidding games in software process education", IEEE Conference on Software Engineering Education and Training, 2013

[PS026]	T. M. Connolly, M. Stansfield and T. Hainey "Applying games-based learning to teach software engineering concepts" Webist 2007 - 3rd International Conference on Web Information Systems and Technologies, 2007

[PS027]	P. Sonchan and S. Ramingwong, "ARMI 2.0: An online risk management simulation" International Conference on Electrical Engineering/Electronics, Computer, Telecommunications and Information Technology, 2015

[PS028] - S. Ramingwong and L. Ramingwong "ARMI: A risk management incorporation", International Conference on Electrical Engineering/Electronics, Computer, Telecommunications and Information Technology, 2014

[PS029]	L. Ganesh, "Board game as a tool to teach software engineering concept - Technical debt", IEEE International Conference on Technology for Education, 2014

[PS030]	A. Calderón and M. Ruiz, "Bringing real-life practice in software project management training through a simulation-based serious game", International Conference on Computer Supported Education, 2014

[PS031]	O. Alsaedi, Z. Toups and J. Cook, "Can a team coordination game help student software project teams?", International Workshop o Cooperative and Human Aspects of Software Engineering, 2016

[PS032]	C. M. Z. Jaramillo, "Communication and traceability game: A way to improve requirements elicitation process teaching", Revista Facultad de Ingenieria, 2010

[PS033]	N. Tillmann, J. de Halleux, T. Xie and J. Bishop, "Constructing Coding Duels in Pex4Fun and Code Hunt", International Symposium on Software Testing and Analysis, 2014

[PS034]	C. G.a Von Wangenheim, R. Savi and A. F.a Borgatto, "DELIVER! - An educational game for teaching Earned Value Management in computing courses", Information and Software Technology, 2012

[PS035]	E. Oh Navarro and A. Van Der Hoek, "Design and evaluation of an educational software process simulation environment and associated model", IEEE Conference on Software Engineering Education and Training, 2005

[PS036]	Y.-W.a Wu, S.-H.b Hsu, S.-L.a Li, W.-H.a Wu and Y.-M.b Huand, "Digital game as a learning approach to enhance practice lesson in software engineering course", International Conference on Computers in Education, 2009

[PS037]	J. Pieper, "Discovering the essence of Software Engineering an integrated game-based approach based on the SEMAT Essence specification", IEEE Global Engineering Education Conference, 2015

[PS038]	T. Xie, N. Tillmann and J. De Halleux, "Educational software engineering: Where software engineering, education, and gaming meet", International Workshop on Games and Software Engineering, 2013

[PS039]	S.-T. Huang, W.-H. Lin and M.-C Hsu, "Embracing business context in pedagogical simulation games - A case with process disciplined project management" IEEE-CS Conference on Software Engineering Education and Training Workshop, 2008

[PS040]	C.a b Gresse Von Wangenheim, M.a Thiry and D.a Kochanski, "Empirical evaluation of an educational game on software measurement", Empirical Software Engineering, 2009

[PS041]	C. Liu and B. Wu, "Enabling collaborative learning with an educational MMORPG", IEEE International Games Innovation Conference, 2011

[PS042]	K. Shaw and J. Dermoudy, "Engendering an Empathy for Software Engineering", Conferences in Research and Practice in Information Technology Series, 2005

[PS043]	E. Ye, C. Liu, J. A. Polack-Wahl, "Enhancing software engineering education using teaching aids in 3-D online virtual worlds", Frontiers In Education Conference, 2007

[PS044]	Y.a Wautelet and M.b Kolp, "E-SPM: An online software project management game", "International Journal of Engineering Education", 2012

[PS045]	C. Szabo, "Evaluating GameDevTycoon for teaching Software Engineering", ACM Technical Symposium on Computer Science Education, 2014

[PS046]	R. Van Solingen, K. Dullemond and B. Van Gameren, "Evaluating the effectiveness of board game usage to teach GSE dynamics", IEEE International Conference on Global Software Engineering, 2011

[PS047]	T. Hainey, T. M. Connolly, M. Stansfield and E. A. Boyle "Evaluation of a game to teach requirements collection and analysis in software engineering at tertiary education level", Computers and Education, 2011

[PS048]	F. B. V. Benitti, and L. Sommariva, "Evaluation of a Game Used to Teach Usability to Undergraduate Students in Computer Science", Journal of Usability Studies, 2015

[PS049]	R. O. Chaves, C. G. Von Wangenheim, J. C. C. Furtado, S. R. B. Oliveira, A. Santos and E. L. Favero, "Experimental evaluation of a serious game for teaching software process modeling", IEEE Transactions on Education, 2015

[PS050]	J. Bergin and F. Grossman, "Extreme construction: Making agile accessible", AGILE Conference, 2006

[PS051]	R. Smith and O. Gotel, "Gameplay to Introduce and Reinforce Requirements Engineering Practices", IEEE International Requirements Engineering Conference, 2008

[PS052]	B. Scharlau, "Games for teaching software development", Annual Conference on Innovation and Technology in Computer Science Education, 2013

[PS053]	J. Beatty and M. Alexander, "Games-Based Requirements Engineering Training: An Initial Experience Report", IEEE International Requirements Engineering Conference, 2008

[PS054]	J. Noll, A. Butterfield, K. Farrell, T. Mason, M. McGuire and R. McKinley, "GSD Sim: A Global Software Development Game", IEEE International Conference on Global Software Engineeering Workshops, 2014

[PS055]	H. Potter, M. Schots, L. Duboc and V. Werneck, "InspectorX: A game for software inspection training and learning", IEEE Conference on Software Engineering Education and Training, 2014

[PS056]	A. Rusu, R. Russell, R. Cocco and S. DiNicolantonio, "Introducing object oriented design patterns through a puzzle-based serious computer game", Frontiers In Education Conference, 2011

[PS057]	D. Carrington, A. Baker and A. van der Hoek, "It's All in the Game: Teaching Software Process Concepts", Frontiers In Education Conference, 2005

[PS058]	J. H. Andrews, "Killer App: A Eurogame about software quality", IEEE Conference on Software Engineering Education and Training, 2013

[PS059]	C. D. C.a De Oliveira, M. E.b Cintra and F. M.b Mendes Neto, "Learning risk management in software projects with a serious game based on intelligent agents and fuzzy systems", Conference of the European Society for Fuzzy Logic and Technology, 2013

[PS060]	S. T. Huang, M. C. Hsu and W. H. Lin, "Management and Education on the Case-Based Complex e-Business Systems Based On Agent Centric Ontology and Simulation Games", IEEE International Conference on e-Business Engineering, 2007

[PS061]	D. Saito, A. Takebayashi and T. Yamaura, "Minecraft-based preparatory training for 
software development project", IEEE International Professional Communication Conference, 2015

[PS062]	M. D. O.b Barros, A. R.a Dantas, G. O.a Veronese and C. M. L.a Werner, "Model-driven 
game development: Experience and model enhancements in software project management education", Software Process Improvement and Practice, 2006

[PS063]	E. Navarro and A. van der Hoek, "Multi-site Evaluation of SimSE", ACM Technical 
Symposium on Computer Science Education, 2009

[PS064]	M. R. Woodward and K. C. Mander, "On Software Engineering Education: Experiences with the Software Hut Game", IEEE Transactions on Education, 1982

[PS065]	G. Jimenez-Diaz, M. Gomez-Albarran and P. A. Gonzalez-Calero, "Pass the ball: Game-based learning of software design", Lecture Notes in Computer Science , 2007

[PS066]	M. J. Lee and A. J. Ko, "Personifying Programming Tool Feedback Improves Novice Programmers' Learning", International Workshop on Computing Education Research, 2011

[PS067]	N. Tillmann, J. De Halleux, T. Xie and J. Bishop, "Pex4Fun: A web-based environment for educational gaming via automated test generation", IEEE/ACM International Conference on Software Engineering, 2013

[PS068]	J. M. Fernandes and S. M. Sousa, "PlayScrum - A card game to learn the scrum agile 
method", International Conference on Games and Virtual Worlds for Serious Applications, 2010 for interactive learning of software project management", Communications in Computer and Information Science, 2015

[PS070]	A. Baker, E. O. Navarroand A. van der Hoek, "Problems and Programmers: An Educational Software Engineering Card Game", International Conference on Software Engineering, 2003

[PS071]	J. E. N. Lino, M. A. Paludo, F. V. Binder, S. Reinehr and A. Malucelli, "Project management game 2D (PMG-2D): A serious game to assist software project managers training", Frontiers In Education Conference, 2015

[PS072]	M. A. Miljanovic and J. S. Bradbury, "Robot on!: A Serious Game for Improving Programming Comprehension", International Workshop on Games and Software Engineering, 2016

[PS073]	G. Jimenez-Diaz, M. Gomez-Albarran and P. A. Gonzalez-Calero, "Role-play virtual environments: Recreational learning of software design", Lecture Notes in Computer Science (including subseries Lecture Notes in Artificial Intelligence and Lecture Notes in Bioinformatics), 2008

[PS074]	T. C. Kohwalter, E. W. G. Clua and L. G. P. Murta, "SDM - An Educational Game for Software Engineering", Brazilian Symposium on Games and Digital Entertainment, 2011

[PS075]	J. Ludewig, Th. Bassler, M. Deininger, K. Schneider and J. Schwille, "SESAM - Simulating software projects", International Conference on Software Engineering and Knowledge Engineering, 1992

[PS076]	C.a Caulfield, S P.a Maj, J. C.b Xia and D.a Veal, "Shall we play a game?", Modern Applied Science, 2012

[PS077]	E. O. Navarro and A. Van Der Hoek, "SimSE: An interactive simulation game for software engineering education", International Conference on Computers and Advanced Technology in Education, 2004

[PS078]	A. Rusu, R. Russell and R. Cocco, "Simulating the software engineering interview process using a decision-based serious computer game", International Conference on Computer Games, 2011

[PS079]	A. Jain and B. Boehm, "SimVBSE: Developing a Game for Value-Based Software Engineering", IEEE Conference on Software Engineering Education and Training, 2006

[PS080]	H. Cervantes, S. Haziyev, O. Hrytsay and R. Kazman, "Smart Decisions: An Architectural Design Game", International Conference on Software Engineering, 2016

[PS081]	M. Latzina and B. Rummel, "Soft(ware) skills in context: corporate usability training aiming at cross-disciplinary collaboration", IEEE Conference on Software Engineering Education and Training, 2003

[PS082]	J. J. Horning and D. B. Wortman, "Software Hut: A Computer Program Engineering Project in the Form of a Game", IEEE Transactions on Software Engineering, 1977

[PS083]	N. Tillmann, J. de Halleux, T. Xie, S. Gulwani and J. Bishop, "Teaching and learning programming and software engineering via interactive gaming", International Conference on Software Engineering, 2013

[PS084]	C. Caulfield, D. Veal and S. P. Maj, "Teaching software engineering project management-A novel approach for software engineering programs", Modern Applied Science, 2011

[PS085]	M.a Ivanovic, Z.a Putnik, Z.a Budimac and K.b Bothe, "Teaching Software Project Management course-Seven years experience", IEEE Global Engineering Education Conference, 2012

[PS086]	M. Paasivaara, V. Heikkila, C. Lassenius and T. Toivola, "Teaching Students Scrum Using LEGO Blocks", International Conference on Software Engineering, 2014

[PS087]	V. T. Heikkila, M; Paasivaara and C.Lassenius, "Teaching University Students Kanban with a Collaborative Board Game", International Conference on Software Engineering, 2016

[PS088]	Kilamo, Terhi	"The Community Game: Learning Open Source Development Through Participatory Exercise", International Academic MindTrek Conference: Envisioning Future Media Environments, 2010

[PS089]	W.-F.a Chen, W.-H.b Wu, T.-Y.c Chuang and P.-N.d Chou, "The effect of varied game-based learning systems in engineering education: An experimental study", International Journal of Engineering Education, 2011

[PS090]	M. D. Ernst and J. Chapin, "The Groupthink Specification Exercise", International Conference on Software Engineering, 2005

[PS091]	A. Wegmann, "Theory and practice behind the course designing enterprisewide IT systems", IEEE Transactions on Education, 2004

[PS092]	E. Oh and A. van der Hoek, "Towards game-based simulation as a method of teaching software engineering", Frontiers In Education Conference, 2002

[PS093]	K.a Vega, H.a Fuks and G.b Carvalho, "Training in requirements by collaboration: Branching stories in second life", Simposio Brasileiro de Sistemas Colaborativos, 2009

[PS094]	G. Taran, "Using Games in Software Engineering Education to Teach Risk Management", IEEE Conference on Software Engineering Education and Training, 2007

[PS095]	G. Jiménez-Díaz, P. González-Calero and M. Gómez-Albarrán, "Using role-play virtual environments to learn software design", European Conference on Games-Based Learning, 2007

[PS096]	A. Zeid, "Using simulation games to teach global software engineering courses", Frontiers In Education Conference, 2015

[PS097]	W. Xu and S. Frezza, "A case study: Integrating a game application-driven approach and social collaborations into software engineering education", International Conference on Enterprise Information Systems, 2011

[PS098]	I. A. Zualkernan, "A course for teaching integrated system design to computer engineering students", IEEE Global Engineering Education Conference, 2014

[PS099]	N. Nitta, Y. Takemura and I. Kume, "A practice of collaborative project-based learning for mutual edification between programming skill and artistic craftsmanship", Frontiers In Education Conference, 2009

[PS100]	A. I. Wang and B. Wu, "An application of a game development framework in higher education", International Journal of Computer Games Technology, 2009

[PS101]	B. Wu, A. I. Wang, J.-E Strom and T. B. Kvamme, "An evaluation of using a Game Development Framework in higher education", IEEE Conference on Software Engineering Education and Training, 2009

[PS102]	U. Wolz and S. M. Pulimood, "An integrated approach to project management through classic CS III and video game development", ACM Technical Symposium on Computer Science Education, 2007

[PS103]	N. Ahmadi and M. Jazayeri, "Analyzing the Learning Process in Online Educational Game Design: A Case Study", Australian Software Engineering Conference, 2014

[PS104]	T. Bay, M. Pedroni and B. Meyer, "By students, for students: A production-quality multimedia library and its application to game-based teaching", Journal of Object Technology, 2008

[PS105]	B. Wu and A. I. Wang, "Comparison of learning software architecture by developing social applications versus games on the android platform", International Journal of Computer Games Technology, 2012

[PS106]	T. Goulding and R. DiTrolio, "Complex Game Development by Freshman Computer Science Majors", ACM Technical Symposium on Computer Science Education, 2007

[PS107]	T. Goulding, "Complex Game Development Throughout the College Curriculum", ACM Technical Symposium on Computer Science Education, 2008

[PS108]	P. V. Gestwicki, "Computer Games As Motivation for Design Patterns", ACM Technical Symposium on Computer Science Education, 2007

[PS109]	V. Isomöttönen and V. Lappalainen, "CS1 with games and an emphasis on TDD and unit testing: Piling a trend upon a trend", ACM Inroads, 2012

[PS110]	C. S. Longstreet and K. Cooper, "Experience report: A sustainable serious educational game capstone project", International Conference on Computer Games, 2013

[PS111]	S. Krusche, B. Reichart, P. Tolstoi and B. Bruegge, "Experiences from an experiential learning course on games development", ACM Technical Symposium on Computer Science Education, 2016

[PS112]	E. Keenan and A. Steele, "Exploring Game Architecture Best-practices with Classic Space Invaders", International Conference on Software Engineering, 2011

[PS113]	B. Wu, A. I. Wang, A. H. Ruud and W. Z. Zhang, "Extending Google Android's application as an educational tool", IEEE International Conference on Digital Game and Intelligent Toy Enhanced Learning, 2010

[PS114]	A. I. Wang, "Extensive Evaluation of Using a Game Project in a Software Architecture Course", ACM Transactions on Computing Education, 2011

[PS115]	H. Schoenau-Fog, L. Reng and L. B. Kofoed, "Fabrication of games and learning: A purposive game production", European Conference on Games-Based Learning, 2015

[PS116]	E. Sweedyk and R. M. Keller, "Fun and Games: A New Software Engineering Course", ACM Technical Symposium on Computer Science Education, 2005

[PS117]	B. Wu and A. I. Wang, "Game development frameworks for SE education", IEEE International Games Innovation Conference, 2011

[PS118]	O.a Denninger and J.b Schimmel, "Game programming and XNA in software engineering education", Computer Games and Allied Technology , 2008

[PS119]	E. Sweedyk, "How middle school teachers solved our SE project problems", IEEE Conference on Software Engineering Education and Training, 2011

[PS120]	S. Acharya and D. Burke, "Incorporating gaming in software engineering projects: Case of RMU monopoly", International Multi-Conference on Society, Cybernetics and Informatics, 2008

[PS121]	R. Burns, L. Pollock and T. Harvey, "Integrating hard and soft skills: Software engineers serving middle school teachers", ACM Technical Symposium on Computer Science Education, 2012

[PS122]	L. B. Sherrell and D. L. Mills, "Introducing Software Engineering Processes via Games and Simulations: A Tri-P-LETS Initiative", Journal of Computing Sciences in Colleges, 2008

[PS123]	M. Yampolsky and W. Scacchi, "Learning Game Design and Software Engineering through a Game Prototyping Experience: A Case Study", International Conference on Software Engineering, 2016

[PS124]	M. S. El-Nasr and B. K. Smith, "Learning through game modding", Computers in Entertainment, 2006

[PS125]	D. Giordano and F. Maiorana, "Object Oriented Design through game development in XNA", Interdisciplinary Engineering Design Education Conference, 2013

[PS126]	Z.a Li, L.b O'Brien, S.c Flint and R.c Sankaranarayana, "Object-oriented Sokoban solver: A serious game project for OOAD and AI education", Frontiers In Education Conference, 2015

[PS127]	A. I. Wang, "Post-mortem analysis of student game projects in a software architecture course: Successes and challenges in student software architecture game projects", International IEEE Consumer Electronics Society's Games Innovations Conference, 2009

[PS128]	T. H. Laine and E. Sutinen, "Refreshing contextualised IT curriculum with a pervasive game project in Tanzania", International Conference on Computing Education Research, 2011

[PS129]	B.a b Maxim, "Serious games as software engineering capstone projects", ASEE Annual Conference and Exposition, 2008

[PS130]	R. Dorner and U. Spierling, "Serious Games Development As a Vehicle for Teaching Entertainment Technology and Interdisciplinary Teamwork: Perspectives and Pitfalls", ACM International Workshop on Serious Games, 2014

[PS131]	R. Kerbs, "Student teamwork: A capstone course in game programming", Frontiers In Education Conference, 2007

[PS132]	T. M. Rao and S. Mitra, "Synergizing AI and OOSE: Enhancing interest in computer science through game-playing and puzzle-solving", AAAI Spring Symposium - Technical Report, 2008

[PS133]	M. A. Gomez-Martin, G. Jimenez-Diaz and J. Arroyo, "Teaching design patterns using a family of games", Conference on Integrating Technology into Computer Science Education, 2009

[PS134]	P. Gestwicki, F.-S Sun and B. Dean, "Teaching Game Design and Game Programming Through Interdisciplinary Courses", Journal of Computing Sciences in Colleges, 2008

[PS135]	P. Gestwicki, "Teaching Game Programming with PlayN", Journal of Computing Sciences in Colleges, 2015

[PS136]	J. E. Sims-Knight and R. L. Upchurch, "Teaching Object-Oriented Design Without Programming: A Progress Report", Computer Science Education, 1993

[PS137]	H. Huni and I. Metz, "Teaching Object-oriented Software Architecture by Example: The Games Factory", ACM SIGPLAN International Conference on Object-Oriented Programming, Systems, Languages, and Applications,  1992

[PS138]	J. Ryoo, F. Fonseca and D. S. Janzen, "Teaching Object-Oriented Software Engineering through Problem-Based Learning in the Context of Game Design", IEEE Conference on Software Engineering Education and Training, 2008

[PS139]	A. Whitley, D. Dave, J. Fenwick, B. T. McDaniel, N. Radford and G. T. Loftis, "Teaching Software Design Principles to Undergraduates by Creating Software Inspired by the Board Game Castle Panic", Journal of Computing Sciences in Colleges, 2014

[PS140]	N. E. Cagiltay, "Teaching software engineering by means of computer-game development: Challenges and opportunities", British Journal of Educational Technology, 2007

[PS141]	M. C. Johnson and Y.-H. Lu, "Teaching software engineering through competition and collaboration", ASEE Annual Conference and Exposition, 2006

[PS142]	K. Claypool and M. Claypool, "Teaching Software Engineering Through Game Design", ACM Technical Symposium on Computer Science Education, 2005

[PS143]	T. W. S. Plum and G. M. Weinberg, "Teaching Structured Programming Attitudes, Even in APL, by Example", ACM Technical Symposium on Computer Science Education, 1974

[PS144]	P. Gestwicki, "The Entity System Architecture and Its Application in an Undergraduate Game Development Studio", International Conference on the Foundations of Digital Games, 2012

[PS145]	Y. Rankin, A. Gooch and B. Gooch, "The Impact of Game Design on Students' Interest in CS", International Conference on Game Development in Computer Science Education, 2008

[PS146]	J. Pirker, A. Kultima and C. Gutl, "The value of game prototyping projects for students and industry", International Conference on Game Jams, Hackathons, and Game Creation Events, 2016

[PS147]	I.a Yoon and E.-Y. E.b Kang, "Transforming experience of computer science software development through developing a usable multiplayer online game in one semester", International Conference on Computer Supported Education, 2014

[PS148]	B. R. Maxim and B. Ridgway, "Use of interdisciplinary teams in game development", Frontiers In Education Conference, 2007

[PS149]	J. C. McKim and H. J. C. Ellis, "Using a multiple term project to teach object oriented programming and design", IEEE Conference on Software Engineering Education and Training, 2004

[PS150]	A. I. Wang and B. Wu, "Using game development to teach software architecture", International Journal of Computer Games Technology, 2011

[PS151]	A. Emam and M. G. Mostafa, "Using game level design as an applied method for Software Engineering education", International Conference on Computer Games, 2012

[PS152]	J. Edgington and S. Leutenegger, "Using the Ancient Game of Rogue in CS1", Journal of Computing Sciences in Colleges, 2008

[PS153]	B. Wu, J. E. Strom, A. I. Wang and T. B. Kvamme, "XQUEST used in software architecture education", International IEEE Consumer Electronics Society's Games Innovations Conference, 2009

[PS154]	S. Sheth, J. Bell and G. Kaiser, "A competitive-collaborative approach for introducing software engineering in a CS2 class", IEEE Conference on Software Engineering Education and Training, 2013

[PS155]	A. Rusu, R. Russell, J. Robinson and A. Rusu, "Learning software engineering basic concepts using a five-phase game", Frontiers In Education Conference, 2010	

[PS156]	O. Shabalina, N. Sadovnikova and A. Kravets, "Methodology of teaching software engineering: Game-based learning cycle", IEEE Eastern European Regional Conference on the Engineering of Computer Based Systems, 2013

