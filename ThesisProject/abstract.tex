Software Engineering (SE) is a discipline that is concerned with the application of theory, knowledge and practice to the effective and efficient development of software systems that meet the user requirements. The Software Engineering curriculum guidelines of the ACM/IEEE and Brazilian Computer Society (SBC) emphasize the necessity of providing students with a good balance between theory and practice, allowing the development of the competences expected for professionals in this area. These curriculum guidelines emphasize that the professional competences emerge through the theoretical study of knowledge units and the practical application of their concepts. However, the nature of the assignments and projects proposed in the classroom are limited in scope and time, increasing the challenge of finding a good balance between theory and practice. One strategy that has been largely used to address this necessity is the introduction of software projects in software engineering education. Besides that, Gamification is a relatively recent trend that has been used in several domains, including SE, to induce certain behavior in people, as well as to improve their motivation and engagement in particular tasks. This thesis project proposes the definition of a framework to support the Gamification of project-based software engineering education. To achieve this goal, a design science research is conducted in order to investigate advantages and shortcomings on the use of Project-Based Learning (PBL) and gamification as pedagogical strategies in software engineering education, and then to propose a reusable framework to support educators and researchers in applying these methods in conjunction. 

\keywords{Software engineering education, project-based learning, gamification}

