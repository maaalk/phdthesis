Reutilização de software é uma estratégia de desenvolvimento em que os componentes de software existentes são utilizados no desenvolvimento de novos sistemas de software. Há muitas vantagens da reutilização no desenvolvimento de software, como a minimização dos esforços de desenvolvimento e melhoria da qualidade de software. %Poucos métodos têm sido propostos na literatura para recomendar oportunidades de reuso. 
Nesta dissertação, é proposto um método para a identificação de oportunidades de reutilização baseados na similaridade dos nomes de dois tipos de entidades orientadas a objetos: classes e métodos. O método desenvolvido, chamado JReuse, computa por meio de uma função de similaridade com o objetivo de identificar classes e métodos de nomes semelhantes, a partir de um conjunto de sistemas de software de um mesmo domínio. Essas classes e métodos compõem um repositório com oportunidades de reutilização. Além disso, apresentamos uma ferramenta protótipo para apoiar o método proposto. O método e a ferramenta foram aplicados em $72$ sistemas de software minerados do GitHub, em 4 domínios diferentes: contabilidade, hospital, restaurante e e-commerce. No total, esses sistemas possuem $1.567.337$ linhas de código, $57.017$ métodos e $12.598$ classes. Depois da sua aplicação, JReuse foi avaliada através de uma pesquisa com $32$ desenvolvedores do GitHub nos domínios avaliados. Como resultado, foi possível obervar que JReuse é capaz de identificar as principais classes e métodos que são mais frequentes em cada domínio selecionado.	

\keywords{Reuso de Software, reuso, artefatos reutilizáveis, oportunidades de reuso, estratégia de extração}
